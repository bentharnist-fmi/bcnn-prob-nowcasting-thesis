\chapter{Conclusions}
\label{chapter:conclusions}


In this work, Bayesian Convolutional Neural Network (BCNN) model variants were formulated and developed in an attempt to produce skillful and reliable ensemble based probabilistic precipitation nowcasts. It was shown that BCNN can be used to produce both deterministic and probabilistic precipitation nowcasts, and that hyperparameters such as the prior distribution posed on weights can have a big impact on the performance of the network and must be chosen wisely. Two facets of BCNN were evaluated. Firstly the skill of its ensemble means used as point estimates for nowcasting precipitations, and secondly the quality of its uncertainty estimates were evaluated to measure the fitness of the model for deterministic and probabilistic nowcasting. These facets were compared to existing baseline models, both extrapolation based and Neural Network based. 

BCNN ensemble averages performed acceptably as deterministic precipitation nowcasts, exhibiting many of the known distinctive features of Neural Network based nowcasts, with both good and bad aspects. Judging from the results, the regularizing effect of the Bayesian approach seems to have worked out to at least some degree. From a probabilistic nowcast point of view on the other hand, the skill of BCNN was poor, characterized by a striking lack of sample diversity, unreliability and low discriminatory skill at all precipitation exceedance probability thresholds studied.

Some parts of the model selection and training processes had their problems, leading to a high margin for improvement capacity with possibly little effort, especially in the realm of making reliable uncertainty estimates. Still, these issues are diverse and have unclear roots, meaning that although they might resolve easily, this is not guaranteed. 

The hopeful prospects in this work are that the developed models are indeed fast at making inference while making great use of the capacity of Neural Networks to fit nonlinear patterns, and most importantly do so while providing an early iteration of uncertainty estimates. The current model has much potential for improvement as outlined in the Discussion Chapter \ref{chapter:discussion}. Although it is not yet ready for operative use, future approaches to probabilistic nowcasting using Neural Networks, regardless of whether or not they are based on this work, may very well attain sufficient proficiency for that purpose, as there is an ever-increasing demand and need for skillful probabilistic precipitation nowcasting methods.