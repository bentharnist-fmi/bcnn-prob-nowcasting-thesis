\chapter{Results}
\label{chapter:results}

\section{Selecting the best BCNN model}

\begin{figure}[h]
	\label{fig:convergence-bcnn}
	\centering
	\missingfigure[figwidth=12cm, figheight=8cm]{Plot showing convergence of different BCNN highlighting model selection process}
	\caption{Plot showing convergence of different BCNN highlighting model selection process}
\end{figure}

\begin{figure}[ht]
	\label{fig:bcnn-candidates}
	\centering
	\missingfigure[figwidth=14cm, figheight=8cm]{Plot showing how different priors with weighting affect an example nowcast}
	\caption{Plot showing how different priors with weighting affect an example nowcast}
\end{figure}

\section{Case studies for nowcasts}
\subsection{Case study 1 : Large-scale Stratiform rain event with convective cells}

30.6.2020 midday ? 

\begin{figure}
	\label{fig:case1}
	\centering
	\subfigure[Prediction mean and uncertainty]{\missingfigure[figwidth=6.5cm, figheight=14cm]{Case 1 mean,std}}
	\subfigure[Exceedance probabilities]{\missingfigure[figwidth=6.5cm, figheight=14cm]{Case 1 mean,std}}
	\caption{case 1}
\end{figure}

\subsection{Case study 2 : Rapidly evolving convective rain event}

15.8.2021 afternoon

\begin{figure}
	\label{fig:case2}
	\centering
	\subfigure[Prediction mean and uncertainty]{\missingfigure[figwidth=6.5cm, figheight=14cm]{Case 1 mean,std}}
	\subfigure[Exceedance probabilities]{\missingfigure[figwidth=6.5cm, figheight=14cm]{Case 1 mean,std}}
	\caption{case 2}
\end{figure}

\section{Deterministic prediction skill (Metrics)}

\begin{figure}
	\label{fig:cont-metrics}
	\centering
	\subfigure[MAE]{
		\missingfigure[figwidth=6cm, figheight=5cm]{}
	}
	\subfigure[ME]{
		\missingfigure[figwidth=6cm, figheight=5cm]{}
	}
	\caption{Continuous Metrics}
\end{figure}

\pagebreak

\begin{figure}
	\label{fig:cat-metrics}
	\centering
	\missingfigure[figwidth=14cm, figheight=12cm]{For 0.5, 5.0, 20.0 mm/h FAR POD CSI ETS}
	\caption{cat metrics}
\end{figure}


\begin{figure}
	\label{fig:fss}
	\centering
	\subfigure[0.5 mm/h]{
		\missingfigure[figwidth=4.2cm, figheight=5cm]{}
	}
	\subfigure[5.0 mm/h]{
		\missingfigure[figwidth=4.2cm, figheight=5cm]{}
	}
	\subfigure[20.0 mm/h]{
		\missingfigure[figwidth=4.2cm, figheight=5cm]{}
	}
	\caption{FSS}
\end{figure}

\begin{figure}
	\label{fig:rapsd}
	\centering
	\subfigure[15 min]{
	\missingfigure[figwidth=4.2cm, figheight=5cm]{}
}
	\subfigure[60 min]{
		\missingfigure[figwidth=4.2cm, figheight=5cm]{}
	}
	\subfigure[180 min]{
		\missingfigure[figwidth=4.2cm, figheight=5cm]{}
	}
	\caption{rapsd}
\end{figure}

\pagebreak
\section{Probabilistic prediction skill (Metrics)}

\begin{figure}
	\label{fig:crps}
	
	\missingfigure[figwidth=4cm, figheight=4cm]{CRPS}
	\caption{CRPS}
\end{figure}

text

\begin{figure}
	\label{fig:prob-metrics}
	
	\caption{roc}
\end{figure}
\missingfigure[figwidth=12cm, figheight=12cm]{ROC curves, reliability diagrams  at 0.5, 5.0, 20.0 mm/h and 1h}

\missingfigure[figwidth=12cm, figheight=6cm]{rankhist at 1h for all 3 models, in 3 subfigures side-by-side}

\missingfigure[figwidth=12cm, figheight=8cm]{Area under the curve for all three models as a function of leadtime (15min, 1h, 2h) and threshold, in one or 2 pics...}
 
