\chapter{Introduction}
\label{chapter:intro}


Nowcasting is defined as weather forecasting at a local scale up to six hours, according the World Meteorological Organization definition of 2010 \cite{schmid2019nowcasting}. The ability to provide an accurate precipitation nowcast has grown during this century into a meteorologically and societally significant challenge. This significance emanates from the fact that early warnings of severe precipitation enable authorities and other actors to make early decisions enabling for example disaster damage control, traffic safety, and in the case of heavy rainfall flash flood prevention, as well as the mitigation of economic loss incurred by them. High-density urbanization has exacerbated these issues, making it evermore vital to discover reliable and skillful methods for precipitation nowcasting. 

Numerical weather prediction (NWP) solves the partial differential equations governing the physical processes of weather and climate to produce forecasts. NWP has seen great improvements over decades, up to the point where it can be used to produce accurate forecasts for up to a week in some cases \cite{bauer_quiet_2015}. However, such method is not applicable to predicting at timescales as short as 0 to 6 hours. This is mainly due to imperfect initialization making simulations unable to reach numerical stability in such short time timescales. 
Other problems with NWP include its computational cost and generally having insufficient spatiotemporal resolution to predict convective events and heavy localized rainfall \cite{schultz_can_2021}. 



To compensate for the shortcomings of NWP in the realm of precipitation nowcasting, many different dedicated models and algorithms have been developed \cite{prudden_review_2020}. Contrary to NWP, these models do not combine data from multiple sources in making predictions, but use only simple input images, and deterministically output their predicted evolution. In practice, many of those systems are based on the estimation of the precipitation advection field from \textit{radar echo image sequences} and the further extrapolation of these sequences along the advection field \cite{prudden_review_2020, rinehart_three-dimensional_1978}. Other methods  have been developed that track and try to nowcast the evolution of individual rain cells instead of the whole grid \cite{prudden_review_2020, dixon1993titan}. 

Dedicated precipitation nowcasting methods have had great success in forecasting the immediate future, but their performance typically degrades quickly, becoming unreliable often in the range of an hour. This is related to the fact that basic extrapolation-based methods have limitations such as failing to capture nonlinear patterns like growth and decay of precipitation, as well as convective cell initiation and their life-cycle. 

A lot of work has been done in nowcasting in trying to incorporate modeling of higher-order and multi-scale phenomena into advection-extrapolation based models. Recently, Deep Learning (DL) based precipitation nowcasting approaches have started showing promising results delving into the issues of traditional methods thanks to the high volume of training data available, the increase in computational power, the expressivity of those models and their relative cheapness of inference \cite{shi_convolutional_2015,shi_deep_2017,ayzel_rainnet_nodate}. The common denominator of all these models is that they are based on Convolutional Neural Networks (CNN), which is a class of DL model basing itself on the inductive bias that in image tasks, it is often sufficient to learn patterns across a small neighborhood around a pixel rather than the entire image, also implying translational invariance, an assumption mostly fulfilled in precipitation nowcasting.


\section{Problem Statement}


While the preceding introduction has cast the problem of nowcasting as finding a single optimal point estimate for future precipitation, it is useful to think about it in a probabilistic way. Because in nowcasting the most interesting phenomena are extreme events, which are those requiring preparation and early warnings, it makes intuitive sense that accurate and reliable probabilistic nowcasts are primordial for operational decision-making in meteorological crises. Also, since smaller spatial scales are less predictable, Adding a degree of uncertainty improves the model usefulness at those scales \cite{germann2002scale}.

Indeed, the interest in probabilistic nowcasts has grown lately and many probabilistic models have been introduced in the last decades \cite{seed_formulation_2013, andersson_model_1991, schmid2002short, fox2005bayesian}, notably the Short-Term Ensemble Prediction System (STEPS) \cite{bowler_steps_2006} and more recently the Lagrangian Integro-Difference equation model with Autoregression (LINDA) \cite{pulkkinen_lagrangian_2021}. These methods produce ensembles, that are used to estimate underlying distributions of data and precipitation probabilities. They perform reasonably well, but  suffer from the same limitations as other extrapolation based methods, namely a limited ability to predict nonlinear patterns of growth and decay. 

So far, only little work has been done on using DL to produce probabilistic precipitation nowcasts, with the biggest breakthrough perhaps being \citet{ravuri_skilful_2021} using adversarially trained deep generative models to produce ensemble nowcasts. There exists many possible ways of making probabilistic nowcasts using deep learning, with most of them focusing on directly modeling probability distributions of data. Another approach, that will be focused on from now on is instead to model the uncertainty of model parameters rather than that of the data, as would be done with STEPS or LINDA. This is done by using Bayesian Neural Networks (BNN) and has an advantage of providing implicit regularization of model parameters thus reducing overfitting, while outputting an ensemble of predictions whose variability in part reflects network parameter uncertainty given the training data. Bayesian neural networks have been proposed for use in other risk-averse application such as biomedical image segmentation \cite{kwon_uncertainty_2020} and autonomous vehicles \cite{mcallister_concrete_2017}. This work is the first attempt to apply such a method to precipitation nowcasting.

%and reliable 

In this work, the problem of making skillful probabilistic precipitation nowcasts is approached by building an uncertainty-aware Neural Network. The first goal is to formulate a Bayesian Convolutional Neural Network (BCNN) model for precipitation nowcasting by turning a baseline CNN into a BNN with optimization over parameters performed using stochastic variational inference (SVI). Secondly, the model is selected and trained using Finnish Meteorological Institute (FMI) radar reflectivity composite images. Finally, various metrics are calculated for BCNN nowcasts in order to assess different aspects of their probabilistic and deterministic predictive skill against baseline models.


\section{Structure of the Thesis}

The present work is organized as follows. Chapter \ref{chapter:background} contains background and a literature review on precipitation nowcasting and Bayesian deep learning, aiming to familiarize the reader with essential concepts regarding the subject. Chapter \ref{chapter:methods} describes the experimental details of the work performed, including the datasets used, the models implemented, as well as verification methods and baselines. 
Chapter \ref{chapter:results} presents the results of the experiments performed, including nowcasting examples of meteorologically interesting events, and calculated metric values compared to the baseline models. Chapter \ref{chapter:discussion} discusses these results, their impact, and their validity in details. Finally, Chapter \ref{chapter:conclusions} closes the thesis by summarizing the most important findings and takeaways. 


